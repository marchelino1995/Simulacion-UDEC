% Exam Template for UMTYMP and Math Department courses
%
% Using Philip Hirschhorn's exam.cls: http://www-math.mit.edu/~psh/#ExamCls
%
% run pdflatex on a finished exam at least three times to do the grading table on front page.
%
%%%%%%%%%%%%%%%%%%%%%%%%%%%%%%%%%%%%%%%%%%%%%%%%%%%%%%%%%%%%%%%%%%%%%%%%%%%%%%%%%%%%%%%%%%%%%%

% These lines can probably stay unchanged, although you can remove the last
% two packages if you're not making pictures with tikz.
\documentclass[11pt]{exam}
\RequirePackage{amssymb, amsfonts, amsmath, latexsym, verbatim, xspace, setspace, blkarray, multirow, array}
\RequirePackage{tikz, pgflibraryplotmarks, pgfplotstable}
\RequirePackage{booktabs,pgfplots,pgfplotstable}

% By default LaTeX uses large margins.  This doesn't work well on exams; problems
% end up in the "middle" of the page, reducing the amount of space for students
% to work on them.
\usepackage[margin=1in]{geometry}


% Here's where you edit the Class, Exam, Date, etc.
\newcommand{\class}{Simulaci\'on}
\newcommand{\code}{}
\newcommand{\term}{Primavera 2018}
\newcommand{\examnum}{Certamen 1}
\newcommand{\examdate}{27/11/18}
\newcommand{\timelimit}{90 Minutos}

% For an exam, single spacing is most appropriate
\singlespacing
% \onehalfspacing
% \doublespacing

% For an exam, we generally want to turn off paragraph indentation
\parindent 0ex

%
\begin{document} 

% These commands set up the running header on the top of the exam pages
\pagestyle{head}
\firstpageheader{}{}{}
\runningheader{\class}{\examnum\ - P\'agina \thepage\ de \numpages}{\examdate}
\runningheadrule

\begin{flushright}
\begin{tabular}{p{2.8in} r l}
\textbf{\class} & \textbf{Nombre:} & \makebox[2in]{\hrulefill}\\
\textbf{\code} && \textbf{C\'odigo de honor:} \\
\textbf{\term} && No he dado ni recibido\\
\textbf{\examnum} && ayuda durante este certamen\\
\textbf{\examdate} && \\
\textbf{Tiempo l\'imite: \timelimit} & \textbf{Firma} & \makebox[2in]{\hrulefill}
\end{tabular}\\
\end{flushright}
\rule[1ex]{\textwidth}{.1pt}


Este certamen contiene \numpages\ p\'aginas (incluyendo esta cubierta) y 
\numquestions\ preguntas.  Cerciorece que su copia contiene todas las p\'aginas.  Ponga su iniciales arriba de cada p\'agina en el caso de que separe las hojas y estas se puedan perder.\\

Usted \textbf{PUEDE} utilizar una hoja A4 escrita en una de sus carillas para el certamen.\\

Se requiere que muestre su trabajo para cada problema en este certamen.  Las siguientes reglas aplican:\\

\begin{minipage}[t]{3.7in}
\vspace{0pt}
\begin{itemize}

\item \textbf{Organize su trabajo}, de forma razonablemente ordenada, en el espacio entregado. Trabajo desorganizado dif\'icil de evaluar recibir\'a poco o nada de puntaje (independiente de su exactitud). 

\item \textbf{Respuestas misteriosas o sin fundamentos no recibir\'an puntaje}.  Una respuesta correcta, sin soporte de calculos, explicaci\'on, o trabajo algebraico \textbf{NO} recibir\'a puntaje; una respuesta incorrecta que sea el resultado de calculos intermedios correctos podr\'ia recibir puntaje parcial.

\item Si necesita mas espacio, use el reverso de la p\'agina; indique claramente cuando haga esto.
\end{itemize}

No escriba en la tabla a la derecha.
\end{minipage}
\hfill
\begin{minipage}[t]{2.3in}
\vspace{0pt}
%\cellwidth{3em}
\gradetablestretch{2}
\vqword{Problem}
\addpoints % required here by exam.cls, even though questions haven't started yet.	
\gradetable[v]%[pages]  % Use [pages] to have grading table by page instead of question

\end{minipage}
\newpage % End of cover page

%%%%%%%%%%%%%%%%%%%%%%%%%%%%%%%%%%%%%%%%%%%%%%%%%%%%%%%%%%%%%%%%%%%%%%%%%%%%%%%%%%%%%
%
% See http://www-math.mit.edu/~psh/#ExamCls for full documentation, but the questions
% below give an idea of how to write questions [with parts] and have the points
% tracked automatically on the cover page.
%
%
%%%%%%%%%%%%%%%%%%%%%%%%%%%%%%%%%%%%%%%%%%%%%%%%%%%%%%%%%%%%%%%%%%%%%%%%%%%%%%%%%%%%%
\begin{questions}
\section*{N\'umeros Aleatorios}
% Basic question
\addpoints
\question Utilizando los par\'ametros $Z_0=1$, $a=630.360.016$, $c=0$, y $m=2.147.483.647$, 
\begin{parts}
\part[9] Genere 3 n\'umeros aleat\'orios utilizando el Generador de Congruencia Lineal explicado en clases.
\fillwithlines{5 in}
\part[6] Usted quiere establecer semillas que estan 100 n\'umeros aparte, determine $Z_{100}$
\fillwithlines{2 in}
\end{parts}


\section*{Variables Aleatorias}
\question Utilizando los n\'umeros aleatorios 0.6754, 0.8602.

\begin{parts}
\part[5] Genere una variable aleatoria uniforme entre 5 y 12 usando el primer n\'umero aleatorio.
\fillwithlines{3.8 in}
\part[5] Genere una variable aleatoria exponencial con $\lambda = 5$ usando el segundo n\'umero aleatorio
\fillwithlines{3.8 in}
\part[5] Utilizando una mezcla de las dos distribuciones anteriores con pesos $p_1 = 0.3$ y $p_2 = 1-p_1$, y con n\'umeros aleatorios 0.1453, 0.8763 genere una variable aleatoria mixta.
\fillwithlines{2 in}
\end{parts}

\section*{Procesos Especiales}

\question Usted necesita determinar los tiempos de llegada de un proceso de intensidad variable en el tiempo. La siguiente informaci\'on ha sido obtenida:

 \[\lambda(x)=
\begin{cases}
3&\text{for $x\in[0,3[$}\\
6&\text{for $x\in[3,5[$}\\
1&\text{for $x\in[5,7]$}\\
\end{cases}
\]

\begin{parts}
\part[10] Determine la distribuci\'on inversa acumulada.
\fillwithlines{4.2 in}
\part[5] Utilizando los siguientes n\'umeros aleatorios: 0.872, 0.145, y 0.543 determine el tiempo de las primeras tres llegadas al sistema.
\fillwithlines{3 in}
\end{parts}

\section*{Comparaci\'on de sistemas alternativos}
%\iffalse
\question Usted desea saber si dos configuraciones de su sistema son estad\'isticamente diferentes y ha recolectado la siguiente informaci\'on.
 \begin{table}[h!]
	\centering
    \setlength{\extrarowheight}{2pt}
    \begin{tabular}{ccccccccccc}
		\toprule
		Experiment & Rep 1& Rep 2& Rep 3& Rep 4& Rep 5\\
		\midrule
			+ + + & 13.4 & 14.78 & 18.52 & 9.56 & 9.92 \\
			+ - + & 7.45 & 9.67 & 8.92 & 5.32 & 6.65 \\
		\bottomrule
    \end{tabular}
  \end{table}
\begin{parts}
\part[10] Determine si las configuraciones producen resultados estad\'isticamente diferentes. Asuma que se utilizaron n\'umeros aleatorios comunes.
\fillwithlines{2.8 in}
\newpage
\fillwithlines{2.8 in}
\part[5] Usted quiere reducir el ancho medio de primer experimento a un 20\% de su valor inical. ?`Cu\'antas muestras adicionales necesita?
\fillwithlines{2 in}

\end{parts}
\end{questions}
\end{document}